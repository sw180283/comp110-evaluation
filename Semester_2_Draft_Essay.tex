\documentclass{scrartcl}

\usepackage[hidelinks]{hyperref}
\usepackage[none]{hyphenat}

\title{Semester 2}
\subtitle{COMP110 - Evaluation}

\author{1503048}

\begin{document}

\maketitle

\section*{Introduction}
The three weaknesses that I can improve are: version control; referencing style and regular programming practice.

\section*{Committing to Version Control Frequently}
My Github commits are often unevenly distributed, for example, a few days with several commits, then no commits for a week.

I tend to work in long bulk sessions and once per sprint, which reduces my professional practice quality. This became an issue after struggling to work regularly on projects and working too long during one programming session.

To improve, I should time-table regular deadlines for commits and schedule my programming sessions around this. Also, I should work regularly, using short sessions to improve concentration and produce higher quality code.

 \begin{itemize}
   \item  S - Make a schedule for expected Github commits each week
	\item M - Set 2 scheduled deadlines for submitting code per week
	\item A - This is a reasonable amount of work when working around other assignments
	\item R - This is relevant to academic practice but also industry practice
	\item T - Each week, for the next month, I should see if I meet the deadlines for each commit
 \end{itemize}

\section*{Referencing Style of Academic Papers}
I struggle with formatting academic citations, in particular, how to reference single pages of a paper or how to cite researchers when named in-line.

This reduces the quality of my academic writing style and referencing, as I often miss opportunities where references could be more specific. This became an issue because I could not find consistent methods for referencing papers, on the internet, for the recommended style.  

To improve, I should ask which method is best to use when referencing, for a specific source.

 \begin{itemize}
   \item  S - Make a referencing guide for a variety of sources
	\item M - Have a complete referencing guide
	\item A - There is plenty of material available online or in books to achieve this action
	\item R - This is relevant to improving my academic writing quality and research practice
	\item T - Next month, I should check to see if I have made a complete referencing guide
 \end{itemize}

\section*{Programming Practice}
Throughout the semester, my programming practice has been restricted to only working on course assignments. I should aspire to work on code outside of this requirement.

I found that without a programming project, my programming practice stopped. This meant that the initial productivity of later projects was low. This became a challenge as I did not set myself any projects between course projects.

To improve, I want to have a project over the summer holidays, and future programming gaps, to work on to maintain programming practice.

 \begin{itemize}
   \item  S \& M - Have a summer game project proposal
	\item A - This is a reasonable aim and should not take much time
	\item R - This is relevant to professional practice
	\item T - Next month, I should check to see if I proposed a summer project
 \end{itemize}
 
 \section*{Conclusion}
I shall feed-forward this evaluation by proposing a game project for the summer to improve my professional practice with strict deadlines for each Github commit to improve my version control. I shall also make a referencing guide to improve my academic writing style.

\end{document}
