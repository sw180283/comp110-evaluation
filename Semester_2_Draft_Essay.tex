\documentclass{scrartcl}

\usepackage[hidelinks]{hyperref}
\usepackage[none]{hyphenat}

\title{Semester 2 Draft Essay}
\subtitle{COMP110 - Evaluation}

\author{1503048}

\begin{document}

\maketitle

\section*{Introduction}
The three weaknesses that I found to improve, during this semester, were: more regular Github commits, better use of referencing academic papers and more regular programming practice.

\section*{Regular Github commits}
\subsection*{Identify}
During my projects, I have found that my Github commits are unevenly distributed with a few days on several commits each, then no commits for a week.

\subsection*{Application, how they affected quality of submission, and why they became challenges}
I only worked in long bulk sessions and once or twice per sprint. This meant that the workflow was poor and reduced my professional practice mark. This became an issue after struggling to work regularly on projects and working too long during one programming session.

\subsection*{To Improve}
To improve, I should time-table regular deadlines for commits and schedule my programming sessions around this. Also I should work regularly, using short sessions to improve concentration and produce higher quality code.

 \begin{itemize}
   \item  S - Make a schedule for expected Github commits each week
	\item M - Set 2 schedule deadlines for submitting code per week
	\item A - This is a reasonable amount of work when working around other assignments
	\item R - This is relevant to academic practice but also industry practice
	\item T - Each week I should see if I meet the deadlines for each commit
 \end{itemize}

\section*{Referencing of Academic Papers}
\subsection*{Identify}
Another weakness I found while working on my essays was that I struggled with proper academic citations. In particular, I did not know how to reference single pages of a paper or how to cite researches when named in-line.

\subsection*{Application, how they affected quality of submission, and why they became challenges}
This affected my mark for academic writing style and referencing, as there was inconsistencies and opportunities where references could have been more specific. This became an issue because I could not find consistent methods for referencing papers, on the internet, for the recommended style.  

\subsection*{To Improve}
To improve, I should ask which method is best to use when referencing, for a specific source if I am struggling.

 \begin{itemize}
   \item  S - Make a referencing guide for a variety of source types
	\item M - Have a complete referencing guide
	\item A - This is a reasonable aim and should not take long to make
	\item R - This is relevant to academic writing and research practice
	\item T - Next month, I should check to see if I have made a complete referencing guide
 \end{itemize}

\section*{Programming Practice}
\subsection*{Identify}
Throughout the semester, my programming practice has been restricted to only working on course assignments. I should be aiming to be working on code outside of this requirement.

\subsection*{Application, how they affected quality of submission, and why they became challenges}
I found that without a programming project, my programming practice stopped. This meant that the initial productivity of later projects was low. This became a challenge as I did not set myself any projects between course projects.

\subsection*{To Improve}
To improve, I want to have a project over the summer holidays, and future programming gaps, to work on to maintain programming practice.

 \begin{itemize}
   \item  S \& M - Have a summer project proposal
	\item A - This is a reasonable aim and should not take much time
	\item R - This is relevant to professional practice
	\item T - Next month, I should check to see if I proposed a summer project
 \end{itemize}
 
 \section*{Conclusion}
 If I can feed-forward this evaluation and improve these key skills, I think that my overall work flow can be improved in addition to my work quality.

\end{document}
